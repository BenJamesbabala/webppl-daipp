%auto-ignore
\section{Background}
\label{sec:background}

\subsection{Probabilistic Programming Basics}

Our system is implemented in the probabilistic programming language WebPPL, which we'll use for examples throughout this paper~\cite{WebPPL}.
WebPPL is a PPL embedded in Javascript.
That is, it adds sampling, condition, and inference operators to a purely-functional subset of JS.
Here's an example program illustrating its basic features:
\begin{lstlisting}
var model = function() {
   var x = sample(Bernoulli({p: 0.75}));
   var mu = a ? 2 : 0;
   observe(Gaussian({mu: mu, sigma: 1}), 1);
   return a;
};

Infer({method: 'mcmc'}, model);
\end{lstlisting}
This program uses MCMC to compute an approximate posterior distribution over the return value of the function \ic{model}. \ic{model} is a simple generative model with one latent Bernoulli variable (\ic{x}) and one observed Gaussian variable, which in this example is observed to have the value \ic{1}. The mean of the observed Gaussian variable (\ic{mu}) is dependent on the value of \ic{x}. Since \ic{model} returns \ic{x}, the result of this program is the posterior marginal distribution over the variable \ic{x}.
\remark{Note that \ic{observe} is built on the more general \ic{factor}?}
In the rest of this paper, we will build on this language to add guide programs and amortized inference to it.

\subsection{Inference as Optimization}

Instead of approximating the posterior $p(\vecstyle{x} | \vecstyle{y})$ with a collection of samples, could instead try to approximate it via a parameterized distribution $\guide_{\vecstyle{y}}(\vecstyle{x} ; \phi)$ which is itself easy to sample from.
This is what variational inference is all about~\cite{VariationalInference}.
The goal is to find parameters $\phi$ such that $\guide_{\vecstyle{y}}(\vecstyle{x} ; \phi)$ is as close as possible to $p(\vecstyle{x} | \vecstyle{y})$, where closeness is typically measured via KL-divergence.
% KL-divergence is the most commonly-used measure of `closeness', leading to the following optimization objective:
% %%%
% \begin{equation*}
% \min_{\phi} \KLD(\guide_{\vecstyle{y}}(\vecstyle{x} ; \phi) || p(\vecstyle{x} | \vecstyle{y}))
% \end{equation*}
% %%%

Need to pick a parametrized family $\guide$ to do this; one common choice is the \emph{mean-field family}:
%%%
\begin{equation*}
\guide^{\textbf{MF}}_{\vecstyle{y}}(\vecstyle{x} ; \phi) = \prod_i g_i(x_i ; \phi_i)
\end{equation*}
%%%
This is a fully-factored distribution: it approximates the posterior as an independent product of marginals $\guide_i$, one for each random variable.
Several existing general-purpose variational inference systems use this scheme~\cite{AVIPP,BBVI}.
This is easy to work with, but it does not capture any of the correlations/dependencies in the true posterior.
This limitation is often acceptable b/c $\guide_{\vecstyle{y}}$ is defined to be specific to a particular $\vecstyle{y}$, and thus the parameters are re-optimized for each new $\vecstyle{y}$.
So this gives us an alternative to Monte Carlo methods (e.g. MCMC) that can be faster and more reliable, but it is still solving inference problems from scratch each time.

\subsection{Amortized Inference}

Remind what amortized inference is: using results of previous inference solutions, or pre-computation in general, to solve later inference problems faster.
Evidence that people do this~\cite{AmortizedInference}.
Some previous work tries to implement this principle for Bayesian networks, by inverting the network topology and learning approximations to the local inverse conditional distributions~\cite{StochasticInverses,NeuralStochasticInverses}.

We can view amortized inference as an extention to the inference as optimization idea.
Instead of defining a parametric family $\guide_{\vecstyle{y}}(\vecstyle{x} ; \phi)$ which is specific to a given $\vecstyle{y}$, we instead define a general family $\guide(\vecstyle{x} | \vecstyle{y} ; \phi)$ which is conditional on $\vecstyle{y}$; that is, it takes $\vecstyle{y}$ as input.
In this setting, mean field no longer applies, as whatever components are used to make up $\guide$ must now be functions of $\vecstyle{y}$.
Can extend mean-field to deal with input data by using neural networks:
%%%
\begin{equation*}
\guide(\vecstyle{x} | \vecstyle{y} ; \phi) = \prod_i \guide_i(x_i ; \text{NN}_i(\vecstyle{y} ; \phi))
\end{equation*}
%%%
That is, the parameters of each local conditional in the guide are computed via a neural network function of $\vecstyle{y}$.
This is amortized inference b/c you can invest time up-front optimizing the weights of these neural networks such that $\guide(\vecstyle{x} | \vecstyle{y} ; \phi)$ is close to $p(\vecstyle{x} | \vecstyle{y})$, and the trained nets can then be used on never-before-seen $\vecstyle{y}$'s.
Several recent approaches to `neural variational inference' use some instantiation of this design pattern~\cite{NVIL,DLGM,AEVB}.

Our system is more general; it allows neural nets also to receive as input any random choices already made:
%%%
\begin{equation*}
\guide(\vecstyle{x} | \vecstyle{y} ; \phi) = \prod_i \guide_i(x_i ; \text{NN}_i(\vecstyle{y}, \vecstyle{x}_{<i} ; \phi))
\end{equation*}
%%%
where $\vecstyle{x}_{<i}$ are the random choices made before choice $i$ is sampled. This allows the networks to capture posterior dependencies between latent variables (which we will demonstrate later in the paper).

\subsection{Model Learning}

This same amortized inference setup can also be used to learn the parameters of generative models. If the generative model $p$ is also parameterized, i.e. $p(\vecstyle{x} | \vecstyle{y} ; \theta)$, then its parameters $\theta$ can be optimized along with the parameters $\phi$ of the guide program.
The `neural variational inference' methods mentioned above can all do this~\cite{NVIL,DLGM,AEVB}, though their formulations of the optimization problem are all slightly different.
In our system, $\theta$ and $\phi$ are treated the same, and we can use the same optimization objective and algorithm whether $p$ has any learnable parameters or not.
We will show examples of how our system makes it easy to use either (regularized) maximum-likelihood learning or full variational Bayesian learning, or to switch between them.

