%auto-ignore
\section{Specifying Guide Programs}
\label{sec:guideSpec}

In this section, we describe our language extensions to WebPPL that allow for the specification of guide programs.
We focus for now on manually-specified guide programs. In Section~\ref{sec:autoGuide}, we build on this interface to automatically derive guide programs.

\subsection{Sampling with Guide Distributions}

Previously, we mentioned that our system restricts guide programs $\guide$ to have the same control flow as the original program $p$, meaning that the guide program samples the same variables in the same order.
Our implementation enforces this restriction by defining the guide program inline with the regular program.
At each \ic{sample} statement, in addition to the distribution that the program $p$ samples from, one can also specify what distribution the guide program $\guide$ should sample from. For example, using the simple program from Section~\ref{sec:pplbasics}:
%%%
\begin{lstlisting}
var model = function() {
   var x = sample(Bernoulli({p: 0.75}), {
      guide: Bernoulli({p: 0.475})
   });
   var mu = x ? 2 : 0;
   observe(Gaussian({mu: mu, sigma: 1}), 0.5);
   return x;
};
\end{lstlisting}
%%%
In this example, the guide samples from a Bernoulli with a different success probability \ic{p}. This particular value happpens to give the true posterior for this program, since this toy problem is simple enough to solve in closed form.
We note that the guide distribution not need be of the same family as the prior distribution; we will see later how this property can be useful.

\subsection{Declaring Optimizable Parameters}

In real problems, we will not know the optimal guides \emph{a prior} and will instead want to learn guides by specifying guide distributions with tunable parameters:
%%%
\begin{lstlisting}
var x = sample(Gaussian({mu: 0, sigma: 1}), {
   guide: Gaussian({mu: paramScalar('guideMu'), sigma: softplus(paramScalar('guideSigma'))})
});
\end{lstlisting}
%%%
Here, \ic{paramScalar(name)} declares an optimizable, real-valued parameter named \ic{name}; there are analogous functions \ic{paramVector}, \ic{paramMatrix}, and \ic{paramTensor} for declaring vector, matrix, and tensor-valued parameters.
Since the standard deviation \ic{sigma} of the Gaussian guide distribution must be positive, we use the \ic{softplus}\footnote{$\text{softplus}(x) = \log(\exp(x) + 1)$} function to map the unbounded value returned by \ic{paramScalar} to $\Reals^{+}$; our system includes similar transforms for parameters with other domains (e.g. \ic{sigmoid} for parameters defined over the interval $[0, 1]$).
Parameters must have unique names so they can be disambiguated by the optimization engine under the hood.

Using variational parameters directly as the guide distribution parameters (as done above) results in a mean field approximation for the variable \ic{x}, as mentioned in Section~\ref{sec:background:variational}.
We can also compute the guide parameters via a neural network:~\remark{Should we cut this out and only show the adnn version?}
%%%
\begin{lstlisting}
// Observed value
var y = 0.5;

// Neural net setup
var nIn = 1;
var nHidden = 3;
var nOut = 2;

var model = function() {
   // Neural net params
   var W1 = paramMatrix(nHidden, nIn, 'W1');
   var b1 = paramVector(nHidden, 'b1');
   var W2 = paramMatrix(nOut, nHidden, 'W2');
   var b3 = paramVector(nOut, 'b2');

   // Use neural net to compute guide params
   var nnInput = Vector([y]);
   var nnOutput = linear(sigmoid(linear(nnInput, W1, b1)), W2, b2);

   var x = sample(Gaussian({mu: 0, sigma: 1}), {
      guide: Gaussian({mu: T.get(nnOutput, 0), sigma: softplus(T.get(nnOutput, 1))})
   });
   observe(Gaussian({mu: x, sigma: 0.5}), y);
   return x;
};
\end{lstlisting}
%%%
Explicitly declaring parameters for and defining the structure of large neural networks can become verbose, so we can instead use the adnn\footnote{\url{https://github.com/dritchie/adnn}} neural net library to include neural nets in our programs:
%%%
\begin{lstlisting}
// Observed value
var y = 0.5;

// Neural net setup
var guideNet = nn.mlp(1, [
   {nOut: 3, activation: nn.sigmoid},
   {nOut: 2}
], 'guideNet');

var model = function() {
   // Use neural net to compute guide params
   var nnInput = Vector([y]);
   var nnOutput = nnEval(guideNet, nnInput);

   var x = sample(Gaussian({mu: 0, sigma: 1}), {
      guide: Gaussian({mu: T.get(nnOutput, 0), sigma: softplus(T.get(nnOutput, 1))})
   });
   observe(Gaussian({mu: x, sigma: 0.5}), y);
   return x;
};
\end{lstlisting}
%%%
In this case, the \ic{guideNet} object has its own parameters, which are registered with the optimization engine when \ic{nnEval} is called.

\subsection{Iterating over Observed Data}

The previous examples have thus far conditioned on a single observation. But real models condition on multiple observations. Our system expresses this pattern with the \ic{mapData} function:
%%%
\begin{lstlisting}
var obs = loadData('data.json');   // List of observations
var guideNet = nn.mlp(1, [
   {nOut: 3, activation: nn.sigmoid},
   {nOut: 2}
], 'guideNet');
var model = function() {
   var mu_x = 0;
   var sigma_x = 1;
   var sigma_y = 0.5;
   var latents = mapData({data: obs, batchSize: 100}, function(y) {
      var nnInput = Vector([y]);
      var nnOutput = nnEval(guideNet, nnInput);
      var x = sample(Gaussian({mu: mu_x, sigma: sigma_x}), {
         guide: Gaussian({mu: T.get(nnOutput, 0), sigma: softplus(T.get(nnOutput, 1))})
      });
      observe(Gaussian({mu: x, sigma: sigma_y}), y);
      return x;
   });
   return latents;
};
\end{lstlisting}
%%%
\ic{mapData} operates much like \ic{map} in a typical functional programming language, but it has two important features: (1) the optimization engine treats every execution of the mapped function as IID, and thus (2) the optimization engine can operate on stochastic mini-batches of the data, sized according to the \ic{batchSize} option.
Property (2) is clearly important for efficient, scalable optimization; we will see in Section~\ref{sec:optimization} how property (1) can also be directly leveraged to improve optimization.


\subsection{Defining Learnable Models}

Thus far we have focused on defining parameterized guides for inference.
We can also use parameters for the generative part of our models as well, making models learnable.
The following three code blocks show possible replacements for line 7 of the previous example, replacing the hardcoded constant \ic{mu_x = 0} with a learnable version:

\lstdefinestyle{learnableModels}{numbers=none,basicstyle=\fontsize{6.5pt}{6.75pt}\selectfont\ttfamily}

\begin{minipage}{0.33\linewidth}
\begin{lstlisting}[style=learnableModels]
// Maximum likelihood
var mu_x = paramScalar('mu_x');
\end{lstlisting}
\end{minipage}
%
\hspace{-2em}
%
\begin{minipage}{0.33\linewidth}
\begin{lstlisting}[style=learnableModels]
// L2-regularized
// maximum likelihood
var mu_x = sample(
   Gaussian({mu: 0, sigma: 1}),
   { guide:
      Delta({v: paramScalar('mu_x')})
   });
\end{lstlisting}
\end{minipage}
%
\hspace{-1em}
%
\begin{minipage}{0.33\linewidth}
\begin{lstlisting}[style=learnableModels]
// Variational Bayes
var mu_x = sample(
   Gaussian({mu: 0, sigma: 1}),
   { guide:
      Gaussian({
        mu: paramScalar('mu_x_m'),
        sigma: softplus(paramScalar('mu_x_s'))
      })
   });
\end{lstlisting}
\end{minipage}

The code in the left block results in maximum likelihood estimation: the inference engine will optimize for the single best parameter value.
The middle code block shows how to achieve regularized maximum likelihood. Here, \ic{mu_x} is given a Gaussian prior. However, the guide distribution is a Delta whose center is given by an optimizable parameter. Thus, the inference engine will still predict a point estiamte for \ic{mu_x}, but the Gaussian prior results in L2 regularization.
Finally, the right block shows a variational Bayesian model: \ic{mu_x} has a Gaussian prior distribution, and the guide samples \ic{mu_x} from an approximate variational Gaussian posterior with optimizable parameters.


\subsection{Examples: Simple Bayesian Networks}

The example code we have built up in this section describes a Bayesian network with one continuous latent variable per continuous observation. Figure~\ref{fig:bn_oneLatent} Top shows the fully assembled code (using maximum likelihood estimation for the generative model parameters), along with a graphical model depiction using the notation of Kigma and Welling~\cite{AEVB}. In this diagram, solid arrows indicate dependencies in the generative model given by the main program, and dashed arrows indicate dependencies in the guide program. $\phi$ is shorthand for all the neural network parameters in the guide program.

Figure~\ref{fig:bn_oneLatent} Bottom shows how to modify this code to instead have one discrete latent variable per observation; this is equivalent to a Gaussian mixture model. In this example, the \ic{simplex} function maps a vector in $\Reals^{n-1}$ to the $n$-dimensional simplex (i.e. a vector whose entries sum to one). This process produces a vector of weights suitable for use as the component probabilities of a discrete random variable.

\lstdefinestyle{smaller}{basicstyle=\fontsize{7pt}{7.5pt}\selectfont\ttfamily}

\begin{figure}

% Continuous
\begin{minipage}{\linewidth}
\begin{minipage}{0.66\linewidth}
\begin{lstlisting}[style=smaller]
var obs = loadData('data.json');
var guideNet = nn.mlp(1, [{nOut: 3, activation: nn.sigmoid}, {nOut: 2}], 'guideNet');
var model = function() {
   var mu_x = paramScalar('mu_x');
   var sigma_x = softplus(paramScalar('sigma_x'));
   var sigma_y = softplus(paramScalar('sigma_y'));
   var latents = mapData({data: obs}, function(y) {
      var nnInput = Vector([y]);
      var nnOutput = nnEval(guideNet, nnInput);
      var x = sample(Gaussian({mu: mu_x, sigma: sigma_x}), {
         guide: Gaussian({mu: T.get(nnOutput, 0),
                          sigma: softplus(T.get(nnOutput, 1))})
      });
      observe(Gaussian({mu: x, sigma: sigma_y}), y);
      return {x: x};
   });
   return latents;
};
\end{lstlisting}
\end{minipage}
%
\begin{minipage}{0.33\linewidth}
\begin{flushright}
\begin{tikzpicture}[scale=1, transform shape]
\node[obs] (y1) {$\mathbf{y}$};
\node[latent, above=of y1] (x1) {$\mathbf{x}$};
\node[const, left=of x1] (phi1) {$\phi$};
\node[const, above right=of x1] (mu_x) {$\mu_{\mathbf{x}}$};
\node[const, right=of x1] (sigma_x) {$\sigma_{\mathbf{x}}$};
\node[const, right=of y1] (sigma_y) {$\sigma_{\mathbf{y}}$};
\edge [dashed] {phi1} {x1};
\edge {mu_x} {x1};
\edge {sigma_x} {x1};
\edge {sigma_y} {y1};
\draw (y1) edge[out=135,in=225,->,dashed] (x1);
\edge {x1} {y1};
\plate [xscale=1.5] {} {(y1)(x1)} {$N$} ;
\end{tikzpicture}
\end{flushright}
\end{minipage}
\end{minipage}

% Discrete
\begin{minipage}{\linewidth}
\begin{minipage}{0.66\linewidth}
\begin{lstlisting}[style=smaller]
var obs = loadData('data.json');
var nComps = 3;
var guideNet = nn.mlp(1, [{nOut: 3, activation: nn.sigmoid}, {nOut: nComps-1}], 'guideNet');
var model = function() {
   var theta_x = simplex(paramVector(nComps-1, 'theta_x'));
   var params_y = [
      {mu: paramScalar('mu_y_1'), sigma: softplus(paramScalar('sigma_y_1'))},
      {mu: paramScalar('mu_y_2'), sigma: softplus(paramScalar('sigma_y_2'))},
      {mu: paramScalar('mu_y_3'), sigma: softplus(paramScalar('sigma_y_3'))}
   ];
   var latents = mapData({data: obs}, function(y) {
      var nnInput = Vector([y]);
      var nnOutput = nnEval(guideNet, nnInput);
      var x = sample(Discrete({ps: theta_x}), {
         guide: Discrete(simplex(nnOutput))
      });
      observe(Gaussian(params_y[x]), y);
      return {x: x};
   });
   return latents;
};
\end{lstlisting}
\end{minipage}
%
\begin{minipage}{0.33\linewidth}
\begin{flushright}
\begin{tikzpicture}[scale=1, transform shape]
\node[obs] (y1) {$\mathbf{y}$};
\node[latent, above=of y1] (x1) {$\mathbf{x}$};
\node[const, left=of x1] (phi1) {$\phi$};
\node[const, right=of x1] (theta_x) {$\theta_{\mathbf{x}}$};
\node[const, above right=of y1] (mu_y) {$\mu^{1\colon3}_{\mathbf{y}}$};
\node[const, right=of y1] (sigma_y) {$\sigma^{1\colon3}_{\mathbf{y}}$};
\edge [dashed] {phi1} {x1};
\edge {theta_x} {x1};
\edge {mu_y} {y1};
\edge {sigma_y} {y1};
\draw (y1) edge[out=135,in=225,->,dashed] (x1);
\edge {x1} {y1};
\plate [xscale=1.5] {} {(y1)(x1)} {$N$} ;
\end{tikzpicture}
\end{flushright}
\end{minipage}
\end{minipage}

\caption{WebPPL code and corresponding graphical models for simple Bayesian networks with one latent variable per observation. \emph{Top:} Continuous latent variable. \emph{Bottom:} Discrete latent variable with 3 discrete values (i.e. a Gaussian mixture model with 3 mixture components).}
\label{fig:bn_oneLatent}

\end{figure}


Figure~\ref{fig:bn_twoLatent} Top shows a slightly more complex Bayesian network with two latent continuous variables. Note that the guide program in this example predicts the two latent variables independently given the observation $\vecstyle{y}$. In Figure~\ref{fig:bn_twoLatent} Bottom, we make some small changes to the code (lines 3 and 17, highlighted in green) to instead have the guide program predict the second latent variable $\vecstyle{x}_2$ as a function of the first latent variable $\vecstyle{x}_1$. This small change allows the guide to capture a posterior dependency that was ingored by the first version.
\remark{We'll show later how this gives better results?}


\begin{figure}

% Two continuous; independent
\begin{minipage}{\linewidth}
\begin{minipage}{0.66\linewidth}
\begin{lstlisting}[style=smaller]
var obs = loadData('data.json');
var guideNet1 = nn.mlp(1, [{nOut: 3, activation: nn.sigmoid}, {nOut: 2}], 'guideNet1');
var guideNet2 = nn.mlp(1, [{nOut: 3, activation: nn.sigmoid}, {nOut: 2}], 'guideNet2');
var model = function() {
   var mu_x1 = paramScalar('mu_x1');
   var sigma_x1 = softplus(paramScalar('sigma_x1'));
   var mu_x2 = paramScalar('mu_x2');
   var sigma_x2 = softplus(paramScalar('sigma_x2'));
   var sigma_y = softplus(paramScalar('sigma_y'));
   var latents = mapData({data: obs}, function(y) {
      var nnInput1 = Vector([y]);
      var nnOutput1 = nnEval(guideNet1, nnInput1);
      var x1 = sample(Gaussian({mu: mu_x1, sigma: sigma_x1}), {
         guide: Gaussian({mu: T.get(nnOutput1, 0),
                          sigma: softplus(T.get(nnOutput1, 1))})
      });
      var nnInput2 = Vector([y]);
      var nnOutput2 = nnEval(guideNet2, nnInput2);
      var x2 = sample(Gaussian({mu: mu_x2, sigma: sigma_x2}), {
         guide: Gaussian({mu: T.get(nnOutput2, 0),
                          sigma: softplus(T.get(nnOutput2, 1))})
      });
      observe(Gaussian({mu: x1 + x2, sigma: sigma_y}), y);
      return {x1: x1, x2: x2};
   });
   return latents;
};
\end{lstlisting}
\end{minipage}
%
\begin{minipage}{0.33\linewidth}
\begin{flushright}
\begin{tikzpicture}[scale=1, transform shape]
\node[obs] (y1) {$\mathbf{y}$};
\node[latent, above left=of y1] (x1) {$\mathbf{x_1}$};
\node[latent, above right=of y1] (x2) {$\mathbf{x_2}$};
\node[const, above right=of x1] (phi1) {$\phi$};
\node[const, above left=of x1] (mu_x1) {$\mu_{\mathbf{x_1}}$};
\node[const, above=of x1] (sigma_x1) {$\sigma_{\mathbf{x_1}}$};
\node[const, above=of x2] (mu_x2) {$\mu_{\mathbf{x_2}}$};
\node[const, above right=of x2] (sigma_x2) {$\sigma_{\mathbf{x_2}}$};
\node[const, right=of y1] (sigma_y) {$\sigma_{\mathbf{y}}$};
\edge [dashed] {phi1} {x1};
\edge [dashed] {phi1} {x2};
\edge {mu_x1} {x1};
\edge {sigma_x1} {x1};
\edge {mu_x2} {x2};
\edge {sigma_x2} {x2};
\edge {sigma_y} {y1};
\draw (y1) edge[out=150,in=270,->,dashed] (x1);
\draw (y1) edge[out=30,in=270,->,dashed] (x2);
\edge {x1} {y1};
\edge {x2} {y1};
\plate [xscale=1.5] {} {(y1)(x1)(x2)} {$N$} ;
\end{tikzpicture}
\end{flushright}
\end{minipage}
\end{minipage}

% Two continuous; dependent
\begin{minipage}{\linewidth}
\begin{minipage}{0.66\linewidth}
\begin{lstlisting}[style=smaller]
var obs = loadData('data.json');
var guideNet1 = nn.mlp(1, [{nOut: 3, activation: nn.sigmoid}, {nOut: 2}], 'guideNet1');
var guideNet2 = nn.mlp(<@\hilite{2}@>, [{nOut: 3, activation: nn.sigmoid}, {nOut: 2}], 'guideNet2');
var model = function() {
   var mu_x1 = paramScalar('mu_x1');
   var sigma_x1 = softplus(paramScalar('sigma_x1'));
   var mu_x2 = paramScalar('mu_x2');
   var sigma_x2 = softplus(paramScalar('sigma_x2'));
   var sigma_y = softplus(paramScalar('sigma_y'));
   var latents = mapData({data: obs}, function(y) {
      var nnInput1 = Vector([y]);
      var nnOutput1 = nnEval(guideNet1, nnInput1);
      var x1 = sample(Gaussian({mu: mu_x1, sigma: sigma_x1}), {
         guide: Gaussian({mu: T.get(nnOutput1, 0),
                          sigma: softplus(T.get(nnOutput1, 1))})
      });
      var nnInput2 = <@\hilite{Vector([y, x1]);}@>
      var nnOutput2 = nnEval(guideNet2, nnInput2);
      var x2 = sample(Gaussian({mu: mu_x2, sigma: sigma_x2}), {
         guide: Gaussian({mu: T.get(nnOutput2, 0),
                          sigma: softplus(T.get(nnOutput2, 1))})
      });
      observe(Gaussian({mu: x1 + x2, sigma: sigma_y}), y);
      return {x1: x1, x2: x2};
   });
   return latents;
};
\end{lstlisting}
\end{minipage}
%
\begin{minipage}{0.33\linewidth}
\begin{flushright}
\begin{tikzpicture}[scale=1, transform shape]
\node[obs] (y1) {$\mathbf{y}$};
\node[latent, above left=of y1] (x1) {$\mathbf{x_1}$};
\node[latent, above right=of y1] (x2) {$\mathbf{x_2}$};
\node[const, above right=of x1] (phi1) {$\phi$};
\node[const, above left=of x1] (mu_x1) {$\mu_{\mathbf{x_1}}$};
\node[const, above=of x1] (sigma_x1) {$\sigma_{\mathbf{x_1}}$};
\node[const, above=of x2] (mu_x2) {$\mu_{\mathbf{x_2}}$};
\node[const, above right=of x2] (sigma_x2) {$\sigma_{\mathbf{x_2}}$};
\node[const, right=of y1] (sigma_y) {$\sigma_{\mathbf{y}}$};
\edge [dashed] {phi1} {x1};
\edge [dashed] {phi1} {x2};
\edge {mu_x1} {x1};
\edge {sigma_x1} {x1};
\edge {mu_x2} {x2};
\edge {sigma_x2} {x2};
\edge {sigma_y} {y1};
\draw (y1) edge[out=150,in=270,->,dashed] (x1);
\draw (y1) edge[out=30,in=270,->,dashed] (x2);
\edge [dashed] {x1} {x2};
\edge {x1} {y1};
\edge {x2} {y1};
\plate [xscale=1.5] {} {(y1)(x1)(x2)} {$N$} ;
\end{tikzpicture}
\end{flushright}
\end{minipage}
\end{minipage}

\caption{WebPPL code and corresponding graphical models for simple Bayesian networks with two latent variables per observation. \emph{Top:} Guide program predicts the two latents independently. \emph{Bottom:} Changing the guide program to treat the second latent variable as conditional on the first (green highlights show changes to the code).}
\label{fig:bn_twoLatent}
\end{figure}

